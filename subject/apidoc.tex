% This file was generated by stechec2-generator. DO NOT EDIT.

\noindent \begin{tabular}{lp{11cm}}
\textbf{Constante:} & HAUTEUR \\
\textbf{Valeur:} & 40 \\
\textbf{Description:} & Nombre de lignes dans la carte \\
\end{tabular}
\vspace{0.2cm} \\

\noindent \begin{tabular}{lp{11cm}}
\textbf{Constante:} & LARGEUR \\
\textbf{Valeur:} & 40 \\
\textbf{Description:} & Nombre de colonnes dans la carte \\
\end{tabular}
\vspace{0.2cm} \\

\noindent \begin{tabular}{lp{11cm}}
\textbf{Constante:} & NB\_TOURS \\
\textbf{Valeur:} & 400 \\
\textbf{Description:} & Nombre de tours à jouer avant la fin de la partie \\
\end{tabular}
\vspace{0.2cm} \\

\noindent \begin{tabular}{lp{11cm}}
\textbf{Constante:} & TAILLE\_DEPART \\
\textbf{Valeur:} & 5 \\
\textbf{Description:} & Taille de départ d'une troupe \\
\end{tabular}
\vspace{0.2cm} \\

\noindent \begin{tabular}{lp{11cm}}
\textbf{Constante:} & TAILLE\_MIN \\
\textbf{Valeur:} & 3 \\
\textbf{Description:} & Taille minimale qu'une troupe peut avoir avant de se disperser \\
\end{tabular}
\vspace{0.2cm} \\

\noindent \begin{tabular}{lp{11cm}}
\textbf{Constante:} & NB\_TROUPES \\
\textbf{Valeur:} & 2 \\
\textbf{Description:} & Nombre de troupes que chaque joueur controle \\
\end{tabular}
\vspace{0.2cm} \\

\noindent \begin{tabular}{lp{11cm}}
\textbf{Constante:} & INTERVALLE\_DISTRIB \\
\textbf{Valeur:} & 5 \\
\textbf{Description:} & Intervalle de distribution de pains par les papys \\
\end{tabular}
\vspace{0.2cm} \\

\noindent \begin{tabular}{lp{11cm}}
\textbf{Constante:} & FREQ\_TUNNEL \\
\textbf{Valeur:} & 1 \\
\textbf{Description:} & Nombre de tunnels qu'un joueur peut creuser par tour \\
\end{tabular}
\vspace{0.2cm} \\

\noindent \begin{tabular}{lp{11cm}}
\textbf{Constante:} & PTS\_ACTION \\
\textbf{Valeur:} & 5 \\
\textbf{Description:} & Nombre de déplacements que peut faire une troupe en un tour \\
\end{tabular}
\vspace{0.2cm} \\

\noindent \begin{tabular}{lp{11cm}}
\textbf{Constante:} & COUT\_CROISSANCE \\
\textbf{Valeur:} & 3 \\
\textbf{Description:} & Nombre de points de mouvement requis pour incrémenter la taille \\
\end{tabular}
\vspace{0.2cm} \\

\noindent \begin{tabular}{lp{11cm}}
\textbf{Constante:} & COUT\_BUISSON \\
\textbf{Valeur:} & 3 \\
\textbf{Description:} & Coût en score de la pose de buisson \\
\end{tabular}
\vspace{0.2cm} \\

\noindent \begin{tabular}{lp{11cm}}
\textbf{Constante:} & ROUND\_FERMETURE \\
\textbf{Valeur:} & 99 \\
\textbf{Description:} & Round à la fin duquel les barrières s'ouvrent ou se ferment \\
\end{tabular}
\vspace{0.2cm} \\


\functitle{erreur} \\
\noindent
\begin{tabular}[t]{@{\extracolsep{0pt}}>{\bfseries}lp{10cm}}
Description~: & Erreurs possibles après avoir effectué une action \\
Valeurs~: &
\small
\begin{tabular}[t]{@{\extracolsep{0pt}}lp{7cm}}
    \textsl{OK}~: & L'action a été effectuée avec succès \\
    \textsl{TROUPE\_INVALIDE}~: & Mauvais identifiant de troupe \\
    \textsl{HORS\_TOUR}~: & Aucune action n'est possible hors de joueur\_tour \\
    \textsl{MOUVEMENTS\_INSUFFISANTS}~: & Il ne reste plus assez de points de mouvements pour effectuer l'action demandée \\
    \textsl{TROP\_GRANDI}~: & La troupe a déjà trop grandi pendant le tour \\
    \textsl{TROP\_CREUSE}~: & Trop de trous ont déjà été creusés pendant le tour \\
    \textsl{NON\_CREUSABLE}~: & Il n'est pas possible de creuser à la position demandée \\
    \textsl{NON\_CONSTRUCTIBLE}~: & La zone demandée n'est pas constructible \\
    \textsl{SCORE\_INSUFFISANT}~: & Le joueur n'a pas assez de points pour construire un buisson \\
    \textsl{POSITION\_INVALIDE}~: & La position demandée est hors du parc \\
    \textsl{DIRECTION\_INVALIDE}~: & La direction spécifiée n'existe pas. \\
    \textsl{PIGEON\_INVALIDE}~: & Le pigeon spécifié n'existe pas. \\
\end{tabular} \\
\end{tabular}

\functitle{direction} \\
\noindent
\begin{tabular}[t]{@{\extracolsep{0pt}}>{\bfseries}lp{10cm}}
Description~: & Directions possibles \\
Valeurs~: &
\small
\begin{tabular}[t]{@{\extracolsep{0pt}}lp{7cm}}
    \textsl{NORD}~: & Sens positif pour les lignes \\
    \textsl{SUD}~: & Sens négatif pour les lignes \\
    \textsl{EST}~: & Sens positif pour les colonnes \\
    \textsl{OUEST}~: & Sens négatif pour les colonnes \\
    \textsl{HAUT}~: & Sens positif pour le niveau \\
    \textsl{BAS}~: & Sens négatif pour le niveau \\
\end{tabular} \\
\end{tabular}

\functitle{type\_case} \\
\noindent
\begin{tabular}[t]{@{\extracolsep{0pt}}>{\bfseries}lp{10cm}}
Description~: & Type de l'élément présent sur une case \\
Valeurs~: &
\small
\begin{tabular}[t]{@{\extracolsep{0pt}}lp{7cm}}
    \textsl{GAZON}~: & Absence d'élément \\
    \textsl{BUISSON}~: & Obstacle impossible à traverser \\
    \textsl{BARRIERE}~: & Élément pouvant être ouvert ou fermé. Une barrière fermée est infranchissable alors qu'une barrière ouverte est analogue à une case vide \\
    \textsl{NID}~: & Élément traversable permettant à la troupe de déposer son inventaire en échange de points \\
    \textsl{PAPY}~: & Élément traversable générant de manière périodique des miches de pain \\
    \textsl{TROU}~: & Interface entre le niveau principal est le niveau souterrain \\
    \textsl{TUNNEL}~: & Bloc du souterrain ayant été creusé \\
    \textsl{TERRE}~: & Bloc du souterrain n'ayant pas encore été creusé \\
\end{tabular} \\
\end{tabular}

\functitle{etat\_barriere} \\
\noindent
\begin{tabular}[t]{@{\extracolsep{0pt}}>{\bfseries}lp{10cm}}
Description~: & État d'une barrière, soit ouvert, soit fermé, soit non-applicable \\
Valeurs~: &
\small
\begin{tabular}[t]{@{\extracolsep{0pt}}lp{7cm}}
    \textsl{OUVERTE}~: & La barrière est ouverte \\
    \textsl{FERMEE}~: & La barrière est fermée \\
    \textsl{PAS\_DE\_BARRIERE}~: & L'élément dont on requiert l'état n'est pas une barrière \\
\end{tabular} \\
\end{tabular}

\functitle{etat\_nid} \\
\noindent
\begin{tabular}[t]{@{\extracolsep{0pt}}>{\bfseries}lp{10cm}}
Description~: & Joueur auquel appartient un nid \\
Valeurs~: &
\small
\begin{tabular}[t]{@{\extracolsep{0pt}}lp{7cm}}
    \textsl{LIBRE}~: & Le nid n'a pas été attribué \\
    \textsl{JOUEUR\_0}~: & Joueur 0 \\
    \textsl{JOUEUR\_1}~: & Joueur 1 \\
    \textsl{PAS\_DE\_NID}~: & L'élément dont on requiert l'état n'est pas un nid \\
\end{tabular} \\
\end{tabular}

\functitle{pigeon\_debug} \\
\noindent
\begin{tabular}[t]{@{\extracolsep{0pt}}>{\bfseries}lp{10cm}}
Description~: & Type de pigeon de debug \\
Valeurs~: &
\small
\begin{tabular}[t]{@{\extracolsep{0pt}}lp{7cm}}
    \textsl{PAS\_DE\_PIGEON}~: & Aucun pigeon, enlève le pigeon présent \\
    \textsl{PIGEON\_BLEU}~: & Pigeon bleu \\
    \textsl{PIGEON\_JAUNE}~: & Pigeon jaune \\
    \textsl{PIGEON\_ROUGE}~: & Pigeon rouge \\
\end{tabular} \\
\end{tabular}

\functitle{type\_action} \\
\noindent
\begin{tabular}[t]{@{\extracolsep{0pt}}>{\bfseries}lp{10cm}}
Description~: & Types d'actions \\
Valeurs~: &
\small
\begin{tabular}[t]{@{\extracolsep{0pt}}lp{7cm}}
    \textsl{ACTION\_AVANCER}~: & Action ``avancer`` \\
    \textsl{ACTION\_GRANDIR}~: & Action ``grandir`` \\
    \textsl{ACTION\_CONSTRUIRE}~: & Action ``construire buisson`` \\
    \textsl{ACTION\_CREUSER}~: & Action ``creuser tunnel`` \\
\end{tabular} \\
\end{tabular}



\functitle{position}
\begin{lst-c++}
struct position {
    int colonne;
    int ligne;
    int niveau;
};
\end{lst-c++}
\noindent
\begin{tabular}[t]{@{\extracolsep{0pt}}>{\bfseries}lp{10cm}}
Description~: & Position dans la carte, donnée par trois coordonnées \\
Champs~: &
\small
\begin{tabular}[t]{@{\extracolsep{0pt}}lp{7cm}}
    \textsl{colonne}~: & Abscisse \\
    \textsl{ligne}~: & Ordonnée \\
    \textsl{niveau}~: & Niveau \\
\end{tabular} \\
\end{tabular}

\functitle{troupe}
\begin{lst-c++}
struct troupe {
    position maman;
    position array canards;
    int taille;
    direction dir;
    int inventaire;
    int pts\_action;
    int id;
};
\end{lst-c++}
\noindent
\begin{tabular}[t]{@{\extracolsep{0pt}}>{\bfseries}lp{10cm}}
Description~: & Une troupe, composée de la maman canard et de ses canetons \\
Champs~: &
\small
\begin{tabular}[t]{@{\extracolsep{0pt}}lp{7cm}}
    \textsl{maman}~: & Position de la maman canard \\
    \textsl{canards}~: & Position des différents canards de la troupe, incluant la maman en première position \\
    \textsl{taille}~: & Taille de la troupe \\
    \textsl{dir}~: & Direction de la troupe \\
    \textsl{inventaire}~: & Nombre de pains de la troupe \\
    \textsl{pts\_action}~: & Nombre de points d'action de la troupe \\
    \textsl{id}~: & Identifiant de la troupe \\
\end{tabular} \\
\end{tabular}

\functitle{etat\_case}
\begin{lst-c++}
struct etat\_case {
    position pos;
    type\_case contenu;
    bool est\_constructible;
    int nb\_pains;
};
\end{lst-c++}
\noindent
\begin{tabular}[t]{@{\extracolsep{0pt}}>{\bfseries}lp{10cm}}
Description~: & Élément constituant le parc \\
Champs~: &
\small
\begin{tabular}[t]{@{\extracolsep{0pt}}lp{7cm}}
    \textsl{pos}~: & Position de la case. Le niveau vaut nécessairement 0 \\
    \textsl{contenu}~: & Type de la case \\
    \textsl{est\_constructible}~: & La case est constructible \\
    \textsl{nb\_pains}~: & Nombre de pains contenus sur la case \\
\end{tabular} \\
\end{tabular}

\functitle{action\_hist}
\begin{lst-c++}
struct action\_hist {
    type\_action action\_type;
    int troupe\_id;
    direction action\_dir;
    position action\_pos;
};
\end{lst-c++}
\noindent
\begin{tabular}[t]{@{\extracolsep{0pt}}>{\bfseries}lp{10cm}}
Description~: & Action représentée dans l'historique \\
Champs~: &
\small
\begin{tabular}[t]{@{\extracolsep{0pt}}lp{7cm}}
    \textsl{action\_type}~: & Type de l'action \\
    \textsl{troupe\_id}~: & Identifiant de la troupe \\
    \textsl{action\_dir}~: & Direction de l'action \\
    \textsl{action\_pos}~: & Position de l'action \\
\end{tabular} \\
\end{tabular}



\begin{minipage}{\linewidth}
\functitle{avancer}
\begin{lst-c++}
erreur avancer(int id, direction dir)
\end{lst-c++}
\noindent
\begin{tabular}[t]{@{\extracolsep{0pt}}>{\bfseries}lp{10cm}}
Description~: & La troupe avance d'une case vers une direction donnée \\
Paramètres~: &
\begin{tabular}[t]{@{\extracolsep{0pt}}ll}
    \textsl{id}~: & Identifiant de la troupe à avancer \\
    \textsl{dir}~: & Direction vers laquelle avancer \\
  \end{tabular} \\
\end{tabular} \\[0.3cm]
\end{minipage}

\begin{minipage}{\linewidth}
\functitle{grandir}
\begin{lst-c++}
erreur grandir(int id)
\end{lst-c++}
\noindent
\begin{tabular}[t]{@{\extracolsep{0pt}}>{\bfseries}lp{10cm}}
Description~: & La troupe grandit \\
Paramètres~: &
\begin{tabular}[t]{@{\extracolsep{0pt}}ll}
    \textsl{id}~: & Identifiant de la troupe à faire grandir \\
  \end{tabular} \\
\end{tabular} \\[0.3cm]
\end{minipage}

\begin{minipage}{\linewidth}
\functitle{construire\_buisson}
\begin{lst-c++}
erreur construire\_buisson(position pos)
\end{lst-c++}
\noindent
\begin{tabular}[t]{@{\extracolsep{0pt}}>{\bfseries}lp{10cm}}
Description~: & Construit un buisson à la position donnée \\
Paramètres~: &
\begin{tabular}[t]{@{\extracolsep{0pt}}ll}
    \textsl{pos}~: & Position où construire le buisson \\
  \end{tabular} \\
\end{tabular} \\[0.3cm]
\end{minipage}

\begin{minipage}{\linewidth}
\functitle{creuser\_tunnel}
\begin{lst-c++}
erreur creuser\_tunnel(position pos)
\end{lst-c++}
\noindent
\begin{tabular}[t]{@{\extracolsep{0pt}}>{\bfseries}lp{10cm}}
Description~: & Creuse un tunnel à la position donnée \\
Paramètres~: &
\begin{tabular}[t]{@{\extracolsep{0pt}}ll}
    \textsl{pos}~: & Position de la case à creuser \\
  \end{tabular} \\
\end{tabular} \\[0.3cm]
\end{minipage}

\begin{minipage}{\linewidth}
\functitle{info\_case}
\begin{lst-c++}
etat\_case info\_case(position pos)
\end{lst-c++}
\noindent
\begin{tabular}[t]{@{\extracolsep{0pt}}>{\bfseries}lp{10cm}}
Description~: & Renvoie les informations concernant une case \\
Paramètres~: &
\begin{tabular}[t]{@{\extracolsep{0pt}}ll}
    \textsl{pos}~: & Position de la case \\
  \end{tabular} \\
\end{tabular} \\[0.3cm]
\end{minipage}

\begin{minipage}{\linewidth}
\functitle{info\_barriere}
\begin{lst-c++}
etat\_barriere info\_barriere(position pos)
\end{lst-c++}
\noindent
\begin{tabular}[t]{@{\extracolsep{0pt}}>{\bfseries}lp{10cm}}
Description~: & Renvoie les informations d'état d'une barrière \\
Paramètres~: &
\begin{tabular}[t]{@{\extracolsep{0pt}}ll}
    \textsl{pos}~: & Position de la barrière \\
  \end{tabular} \\
\end{tabular} \\[0.3cm]
\end{minipage}

\begin{minipage}{\linewidth}
\functitle{info\_nid}
\begin{lst-c++}
etat\_nid info\_nid(position pos)
\end{lst-c++}
\noindent
\begin{tabular}[t]{@{\extracolsep{0pt}}>{\bfseries}lp{10cm}}
Description~: & Renvoie les informations d'état d'un nid \\
Paramètres~: &
\begin{tabular}[t]{@{\extracolsep{0pt}}ll}
    \textsl{pos}~: & Position du nid \\
  \end{tabular} \\
\end{tabular} \\[0.3cm]
\end{minipage}

\begin{minipage}{\linewidth}
\functitle{papy\_tours\_restants}
\begin{lst-c++}
int papy\_tours\_restants(position pos)
\end{lst-c++}
\noindent
\begin{tabular}[t]{@{\extracolsep{0pt}}>{\bfseries}lp{10cm}}
Description~: & Renvoie le nombre de tours restants avant qu'un papy dépose une miche de pain. Retourne -1 si aucun papy ne se trouve à la position demandée \\
Paramètres~: &
\begin{tabular}[t]{@{\extracolsep{0pt}}ll}
    \textsl{pos}~: & Position du papy \\
  \end{tabular} \\
\end{tabular} \\[0.3cm]
\end{minipage}

\begin{minipage}{\linewidth}
\functitle{troupes\_joueur}
\begin{lst-c++}
troupe array troupes\_joueur(int id\_joueur)
\end{lst-c++}
\noindent
\begin{tabular}[t]{@{\extracolsep{0pt}}>{\bfseries}lp{10cm}}
Description~: & Renvoie les troupes d'un joueur. Si le joueur est invalide, tous les champs valent -1. \\
Paramètres~: &
\begin{tabular}[t]{@{\extracolsep{0pt}}ll}
    \textsl{id\_joueur}~: & Numéro du joueur concerné \\
  \end{tabular} \\
\end{tabular} \\[0.3cm]
\end{minipage}

\begin{minipage}{\linewidth}
\functitle{pains}
\begin{lst-c++}
position array pains()
\end{lst-c++}
\noindent
\begin{tabular}[t]{@{\extracolsep{0pt}}>{\bfseries}lp{10cm}}
Description~: & Renvoie la position des pains récupérables \\
\end{tabular} \\[0.3cm]
\end{minipage}

\begin{minipage}{\linewidth}
\functitle{debug\_poser\_pigeon}
\begin{lst-c++}
erreur debug\_poser\_pigeon(position pos, pigeon\_debug pigeon)
\end{lst-c++}
\noindent
\begin{tabular}[t]{@{\extracolsep{0pt}}>{\bfseries}lp{10cm}}
Description~: & Pose un pigeon de debug sur la case indiquée \\
Paramètres~: &
\begin{tabular}[t]{@{\extracolsep{0pt}}ll}
    \textsl{pos}~: & Case où poser le pigeon \\
    \textsl{pigeon}~: & Pigeon à afficher sur la case \\
  \end{tabular} \\
\end{tabular} \\[0.3cm]
\end{minipage}

\begin{minipage}{\linewidth}
\functitle{historique}
\begin{lst-c++}
action\_hist array historique()
\end{lst-c++}
\noindent
\begin{tabular}[t]{@{\extracolsep{0pt}}>{\bfseries}lp{10cm}}
Description~: & Renvoie la liste des actions effectuées par l'adversaire durant son tour, dans l'ordre chronologique. Les actions de débug n'apparaissent pas dans cette liste. \\
\end{tabular} \\[0.3cm]
\end{minipage}

\begin{minipage}{\linewidth}
\functitle{gain}
\begin{lst-c++}
int gain(int nb\_pains)
\end{lst-c++}
\noindent
\begin{tabular}[t]{@{\extracolsep{0pt}}>{\bfseries}lp{10cm}}
Description~: & Renvoie le gain en score que le nombre de pains passé en entrée rapporterait s'ils étaient tous déposés d'un coup dans un nid \\
Paramètres~: &
\begin{tabular}[t]{@{\extracolsep{0pt}}ll}
    \textsl{nb\_pains}~: & Nombre de miches de pain déposées \\
  \end{tabular} \\
\end{tabular} \\[0.3cm]
\end{minipage}

\begin{minipage}{\linewidth}
\functitle{inventaire}
\begin{lst-c++}
int inventaire(int taille)
\end{lst-c++}
\noindent
\begin{tabular}[t]{@{\extracolsep{0pt}}>{\bfseries}lp{10cm}}
Description~: & Renvoie la taille de l'inventaire d'une troupe de taille donnée \\
Paramètres~: &
\begin{tabular}[t]{@{\extracolsep{0pt}}ll}
    \textsl{taille}~: & Taille de la troupe \\
  \end{tabular} \\
\end{tabular} \\[0.3cm]
\end{minipage}

\begin{minipage}{\linewidth}
\functitle{trouver\_chemin}
\begin{lst-c++}
direction array trouver\_chemin(position depart, position arrivee)
\end{lst-c++}
\noindent
\begin{tabular}[t]{@{\extracolsep{0pt}}>{\bfseries}lp{10cm}}
Description~: & Trouve un plus court chemin ouvert entre deux positions. Renvoie une liste vide si les deux positions sont égales ou si aucun chemin n'existe. \\
Paramètres~: &
\begin{tabular}[t]{@{\extracolsep{0pt}}ll}
    \textsl{depart}~: & Position de départ \\
    \textsl{arrivee}~: & Position d'arrivée \\
  \end{tabular} \\
\end{tabular} \\[0.3cm]
\end{minipage}

\begin{minipage}{\linewidth}
\functitle{moi}
\begin{lst-c++}
int moi()
\end{lst-c++}
\noindent
\begin{tabular}[t]{@{\extracolsep{0pt}}>{\bfseries}lp{10cm}}
Description~: & Renvoie votre numéro de joueur. \\
\end{tabular} \\[0.3cm]
\end{minipage}

\begin{minipage}{\linewidth}
\functitle{adversaire}
\begin{lst-c++}
int adversaire()
\end{lst-c++}
\noindent
\begin{tabular}[t]{@{\extracolsep{0pt}}>{\bfseries}lp{10cm}}
Description~: & Renvoie le numéro du joueur adverse. \\
\end{tabular} \\[0.3cm]
\end{minipage}

\begin{minipage}{\linewidth}
\functitle{score}
\begin{lst-c++}
int score(int id\_joueur)
\end{lst-c++}
\noindent
\begin{tabular}[t]{@{\extracolsep{0pt}}>{\bfseries}lp{10cm}}
Description~: & Renvoie le score du joueur `id\_joueur`. Renvoie -1 si le joueur est invalide. \\
Paramètres~: &
\begin{tabular}[t]{@{\extracolsep{0pt}}ll}
    \textsl{id\_joueur}~: & Numéro du joueur concerné \\
  \end{tabular} \\
\end{tabular} \\[0.3cm]
\end{minipage}

\begin{minipage}{\linewidth}
\functitle{annuler}
\begin{lst-c++}
bool annuler()
\end{lst-c++}
\noindent
\begin{tabular}[t]{@{\extracolsep{0pt}}>{\bfseries}lp{10cm}}
Description~: & Annule la dernière action. Renvoie faux quand il n'y a pas d'action à annuler ce tour-ci \\
\end{tabular} \\[0.3cm]
\end{minipage}

\begin{minipage}{\linewidth}
\functitle{tour\_actuel}
\begin{lst-c++}
int tour\_actuel()
\end{lst-c++}
\noindent
\begin{tabular}[t]{@{\extracolsep{0pt}}>{\bfseries}lp{10cm}}
Description~: & Retourne le numéro du tour actuel. \\
\end{tabular} \\[0.3cm]
\end{minipage}

\begin{minipage}{\linewidth}
\functitle{afficher\_erreur}
\begin{lst-c++}
void afficher\_erreur(erreur v)
\end{lst-c++}
\noindent
\begin{tabular}[t]{@{\extracolsep{0pt}}>{\bfseries}lp{10cm}}
Description~: & Affiche le contenu d'une valeur de type erreur \\
Paramètres~: &
\begin{tabular}[t]{@{\extracolsep{0pt}}ll}
    \textsl{v}~: & The value to display \\
  \end{tabular} \\
\end{tabular} \\[0.3cm]
\end{minipage}

\begin{minipage}{\linewidth}
\functitle{afficher\_direction}
\begin{lst-c++}
void afficher\_direction(direction v)
\end{lst-c++}
\noindent
\begin{tabular}[t]{@{\extracolsep{0pt}}>{\bfseries}lp{10cm}}
Description~: & Affiche le contenu d'une valeur de type direction \\
Paramètres~: &
\begin{tabular}[t]{@{\extracolsep{0pt}}ll}
    \textsl{v}~: & The value to display \\
  \end{tabular} \\
\end{tabular} \\[0.3cm]
\end{minipage}

\begin{minipage}{\linewidth}
\functitle{afficher\_type\_case}
\begin{lst-c++}
void afficher\_type\_case(type\_case v)
\end{lst-c++}
\noindent
\begin{tabular}[t]{@{\extracolsep{0pt}}>{\bfseries}lp{10cm}}
Description~: & Affiche le contenu d'une valeur de type type\_case \\
Paramètres~: &
\begin{tabular}[t]{@{\extracolsep{0pt}}ll}
    \textsl{v}~: & The value to display \\
  \end{tabular} \\
\end{tabular} \\[0.3cm]
\end{minipage}

\begin{minipage}{\linewidth}
\functitle{afficher\_etat\_barriere}
\begin{lst-c++}
void afficher\_etat\_barriere(etat\_barriere v)
\end{lst-c++}
\noindent
\begin{tabular}[t]{@{\extracolsep{0pt}}>{\bfseries}lp{10cm}}
Description~: & Affiche le contenu d'une valeur de type etat\_barriere \\
Paramètres~: &
\begin{tabular}[t]{@{\extracolsep{0pt}}ll}
    \textsl{v}~: & The value to display \\
  \end{tabular} \\
\end{tabular} \\[0.3cm]
\end{minipage}

\begin{minipage}{\linewidth}
\functitle{afficher\_etat\_nid}
\begin{lst-c++}
void afficher\_etat\_nid(etat\_nid v)
\end{lst-c++}
\noindent
\begin{tabular}[t]{@{\extracolsep{0pt}}>{\bfseries}lp{10cm}}
Description~: & Affiche le contenu d'une valeur de type etat\_nid \\
Paramètres~: &
\begin{tabular}[t]{@{\extracolsep{0pt}}ll}
    \textsl{v}~: & The value to display \\
  \end{tabular} \\
\end{tabular} \\[0.3cm]
\end{minipage}

\begin{minipage}{\linewidth}
\functitle{afficher\_pigeon\_debug}
\begin{lst-c++}
void afficher\_pigeon\_debug(pigeon\_debug v)
\end{lst-c++}
\noindent
\begin{tabular}[t]{@{\extracolsep{0pt}}>{\bfseries}lp{10cm}}
Description~: & Affiche le contenu d'une valeur de type pigeon\_debug \\
Paramètres~: &
\begin{tabular}[t]{@{\extracolsep{0pt}}ll}
    \textsl{v}~: & The value to display \\
  \end{tabular} \\
\end{tabular} \\[0.3cm]
\end{minipage}

\begin{minipage}{\linewidth}
\functitle{afficher\_type\_action}
\begin{lst-c++}
void afficher\_type\_action(type\_action v)
\end{lst-c++}
\noindent
\begin{tabular}[t]{@{\extracolsep{0pt}}>{\bfseries}lp{10cm}}
Description~: & Affiche le contenu d'une valeur de type type\_action \\
Paramètres~: &
\begin{tabular}[t]{@{\extracolsep{0pt}}ll}
    \textsl{v}~: & The value to display \\
  \end{tabular} \\
\end{tabular} \\[0.3cm]
\end{minipage}

\begin{minipage}{\linewidth}
\functitle{afficher\_position}
\begin{lst-c++}
void afficher\_position(position v)
\end{lst-c++}
\noindent
\begin{tabular}[t]{@{\extracolsep{0pt}}>{\bfseries}lp{10cm}}
Description~: & Affiche le contenu d'une valeur de type position \\
Paramètres~: &
\begin{tabular}[t]{@{\extracolsep{0pt}}ll}
    \textsl{v}~: & The value to display \\
  \end{tabular} \\
\end{tabular} \\[0.3cm]
\end{minipage}

\begin{minipage}{\linewidth}
\functitle{afficher\_troupe}
\begin{lst-c++}
void afficher\_troupe(troupe v)
\end{lst-c++}
\noindent
\begin{tabular}[t]{@{\extracolsep{0pt}}>{\bfseries}lp{10cm}}
Description~: & Affiche le contenu d'une valeur de type troupe \\
Paramètres~: &
\begin{tabular}[t]{@{\extracolsep{0pt}}ll}
    \textsl{v}~: & The value to display \\
  \end{tabular} \\
\end{tabular} \\[0.3cm]
\end{minipage}

\begin{minipage}{\linewidth}
\functitle{afficher\_etat\_case}
\begin{lst-c++}
void afficher\_etat\_case(etat\_case v)
\end{lst-c++}
\noindent
\begin{tabular}[t]{@{\extracolsep{0pt}}>{\bfseries}lp{10cm}}
Description~: & Affiche le contenu d'une valeur de type etat\_case \\
Paramètres~: &
\begin{tabular}[t]{@{\extracolsep{0pt}}ll}
    \textsl{v}~: & The value to display \\
  \end{tabular} \\
\end{tabular} \\[0.3cm]
\end{minipage}

\begin{minipage}{\linewidth}
\functitle{afficher\_action\_hist}
\begin{lst-c++}
void afficher\_action\_hist(action\_hist v)
\end{lst-c++}
\noindent
\begin{tabular}[t]{@{\extracolsep{0pt}}>{\bfseries}lp{10cm}}
Description~: & Affiche le contenu d'une valeur de type action\_hist \\
Paramètres~: &
\begin{tabular}[t]{@{\extracolsep{0pt}}ll}
    \textsl{v}~: & The value to display \\
  \end{tabular} \\
\end{tabular} \\[0.3cm]
\end{minipage}

